\section{Database}

	Questo database è stato creato usando PostgreSQL in combinazione con l’utilizzo di JDBC per la comunicazione con il server.
	Di seguito andiamo a mostrare lo schema logico del database, il diagramma ER e l’implementazione.
	
\subsection{Schema logico}

	\begin{itemize}
		\item Centro\_vaccinale(\underline{Nome}, tipologia)
		
		\item Cittadino(\underline{CF}, Nome, Cognome, email, Username,Password)
		
		\item Vaccinazione(\underline{IDvacc}, \underline{CF\textsuperscript{citt}}, Data, Vaccino)
		
		\item EventoAvverso(\underline{IDea}, \underline{ID\textsuperscript{vacc}}, Evento, Intensità)
		
		\item Ospedaliero(Nome\textsuperscript{centrivaccinali})
		
		\item Hub(Nome\textsuperscript{centrivaccinali})
	\end{itemize}
	
	In questo schema sono state evidenziate in \textbf{\textit{corsivo grassetto}} le chiavi primarie della tabella e in \underline{\textit{corsivo sottolineato}} le chiavi esterne della tabella.
	
\subsection{Chiavi esterne}
	
	\paragraph{Indirizzo}
	Presenta una chiave esterna denominata \emph{centro\_vaccinale} 

\begin{listing}[t]
	\begin{minted}{sql}
CREATE TABLE centro_vaccinale (
	nome VARCHAR(255),
	tipologia VARCHAR(10),
	PRIMARY KEY (nome)
);
	\end{minted}
\caption{Codice per la creazione della tabella CentriVaccinali}
\end{listing}

\begin{listing}[t]
	\begin{minted}{sql}
CREATE TABLE Indirizzo (
	id_ind SERIAL PRIMARY KEY,
	identificatore VARCHAR(10),
	localizzazione VARCHAR(255),
	civico NUMERIC(4),
	comune VARCHAR(255),
	provincia CHAR(2),
	zip CHAR(5),
	centro_vaccinale VARCHAR(255),
	FOREIGN KEY (centro_vaccinale) REFERENCES centro_vaccinale(nome)
		ON UPDATE CASCADE
		ON DELETE CASCADE
);
	\end{minted}
\caption{Codice per la creazione della tabella Indirizzi}
\end{listing}

\begin{listing}[t]
	\begin{minted}{sql}
CREATE TABLE Registrazione (
	centro_vaccinale VARCHAR(255) REFERENCES centro_vaccinale
		ON UPDATE CASCADE
		ON DELETE CASCADE,
	CF CHAR(16) UNIQUE REFERENCES cittadino
		ON UPDATE CASCADE
		ON DELETE CASCADE,
	codice SERIAL,
	PRIMARY KEY (centro_vaccinale,CF)
);
	\end{minted}
\caption{Codice per la creazione della tabella Registrazione}
\end{listing}

\begin{listing}[t]
	\begin{minted}{sql}
CREATE TABLE cittadino (
	CF CHAR(16),
	nome VARCHAR(128),
	cognome VARCHAR(128),
	email VARCHAR(255),
	password VARCHAR(255),
	username VARCHAR(255) UNIQUE,
	PRIMARY KEY (CF)
);
	\end{minted}
\caption{Codice per la creazione della tabella Cittadini}
\end{listing}

\begin{listing}[t]
	\begin{minted}{sql}
CREATE TABLE Vaccinazione (
	ID_vaccino SERIAL PRIMARY KEY,
	data DATE,
	vaccino VARCHAR(20),
	CF_citt CHAR(16),
	FOREIGN KEY (CF_citt) REFERENCES Cittadino
		ON UPDATE CASCADE
		ON DELETE CASCADE
);
	\end{minted}
\caption{Codice per la creazione della tabella Vaccinazioni}
\end{listing}

\begin{listing}[t]
	\begin{minted}{sql}
CREATE TABLE EventoAvverso (
	ID_ea SERIAL PRIMARY KEY,
	evento VARCHAR(32),
	intensita SMALLINT,
	ID_vaccino NUMERIC(6),
	FOREIGN KEY (ID_vaccino) REFERENCES Vaccinazione(ID_vaccino)
		ON UPDATE CASCADE
		ON DELETE CASCADE
);
	\end{minted}
\caption{Codice per la creazione della tabella EventiAvversi}
\end{listing}