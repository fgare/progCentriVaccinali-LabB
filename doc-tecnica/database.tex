\section{Database}

	Questo database è stato creato usando PostgreSQL in combinazione con l’utilizzo di JDBC per la comunicazione con il server.
	Di seguito andiamo a mostrare lo schema logico del database, il diagramma ER e l’implementazione.
	
\subsection{Schema logico}

	\begin{itemize}
		\item Centro\_vaccinale(\underline{Nome}, tipologia)
		
		\item Cittadino(\underline{CF}, Nome, Cognome, email, Username,Password)
		
		\item Vaccinazione(\underline{IDvacc}, \underline{CF\textsuperscript{citt}}, Data, Vaccino)
		
		\item EventoAvverso(\underline{IDea}, \underline{ID\textsuperscript{vacc}}, Evento, Intensità)
		
		\item Ospedaliero(Nome\textsuperscript{centrivaccinali})
		
		\item Hub(Nome\textsuperscript{centrivaccinali})
	\end{itemize}
	
	In questo schema sono state evidenziate in \textbf{\textit{corsivo grassetto}} le chiavi primarie della tabella e in \underline{\textit{corsivo sottolineato}} le chiavi esterne della tabella.
	
\subsection{Chiavi esterne}
	
	\paragraph{Indirizzo}
	Presenta una chiave esterna denominata \emph{centro\_vaccinale} 
	%continuare...