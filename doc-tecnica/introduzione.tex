\section{Introduzione}

	Tatum Vaccini è un progetto sviluppato nell'ambito del progetto di Laboratorio~A per il corso di laurea in Informatica dell'Università degli Studi dell'Insubria.
	Il progetto è sviluppato in Java 19, usa un’interfaccia grafica costruita con OpenJFx 19ed è stato sviluppato e testato sui sistemi operativi Windows 10, Windows 11 e Mac OS Ventura.\\
	Esso è costituito da 3 applicazioni distinte: \emph{Applicazione Centri Vaccinali}, \emph{Applicazione cittadini} e \emph{Applicazione server}.
	
\subsection{Librerie esterne utilizzate}
	L'unica libreria esterna di cui l'applicazione fa uso è OpenJFx~12, la quale contiene tutti gli elementi necessari allo sviluppo dell'interfaccia grafica.
	Si è reso necessario appoggiarsi a una libreria esterna in quanto, a partire dalla release 9 del linguaggio Java, JavaFx non è più inclusa all'interno del Java Development Kit (JDK) distribuito da Oracle.
	
\subsection{Struttura generale delle classi}
	Il progetto è organizzato in quattro moduli: \emph{modulo server}, \emph{modulo centri vaccinali}, \emph{modulo cittadini} e \emph{modulo common}.
	La struttura generale è riportata in Figura \ref*{graph:gerarchiaClassi}.
	Descriveremo nel seguito le classi di cui si compone ogni modulo e come esse sono organizzate in package.
	
	\begin{figure}[p]

\dirtree{%
	.1 parent.
		%
		.2 ApplicazioneServer.
			.3 DataManager.
			.3 ServerImpl.
			.3 ServerMain.
			.3 SWVar.
			.3 File.
				.4 DBHandler.
		%
		.2 ApplicazioneCV.
			.3 Main.
			.3 NewMain.
			.3 Client.
				.4 ClientCittadinoInterface.
				.4 ClientMedico.
			.3 GUI.
				.4 ControllerCVRegistrato.
				.4 ControllerHomepage.
				.4 ControllerNuovoCV.
				.4 ControllerRegistraVaccinato.
				.4 UniversalMethods.
		%
		.2 AplicazioneCittadini.
			.3 Client.
				.4 ClientCittadino.
			.3 GUI.
				.4 ControllerAccessoCittadino.
				.4 ControllerHomepage.
				.4 ControllerInserimentoEventiAvversi.
				.4 ControllerRegistraCittadino.
				.4 ControllerSceltaRicerca.
				.4 ControllerVisualizzaCVperComuneTipologia.
				.4 ControllerVisualizzaCvPerNome.
				.4 UniversalMethods.
			.3 Main.
			.3 NewMain.
		%
		.2 common.
			.3 CentroVaccinale.
			.3 Cittadino.
			.3 EventoAvverso.
			.3 Indirizzo.
			.3 ServerInterface.
			.3 Vaccinazione.
}

	\caption{Gerarchia delle classi del progetto}
	\label{graph:gerarchiaClassi}
\end{figure}
	
	\paragraph{Applicazione server}
	L'applicazione server rappresenta il cuore dell'intero progetto, ad essa si collegano in remoto le altre applicazioni e gestisce inoltre l'interazione con il DBMS.
	
	
	\paragraph{Applicazione centri vaccinali}
	Il modulo centri vaccinali è la parte del programma dedicata agli operatori sanitari, essa permette di registrare nuovi centri vaccinali ed inserire vaccinazioni.


	\paragraph{Applicazione cittadini}
	Il modulo cittadini è infine l'applicazione orientata all'uso da parte dei pazienti, essa permette la registrazione e l'inserimento di eventuali evento avversi riscontrati in seguito alla vaccinazione.
	
	
	\paragraph{Modulo common}
	Questo modulo ha il solo scopo di fungere da \emph{contenitore} delle classi condivise dall'intero progetto.
	Esso non è pertanto un'applicazione avviabile ma solamente un modulo verso cui tutti gli altri moduli hanno dipendenze.