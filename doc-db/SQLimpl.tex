\section{Implementazione SQL}

\subsection{Chiavi esterne e integrità referenziale}

	\begin{description}
		\item[Indirizzo]
			Presenta una chiave esterna denominata \verb|centro_vaccinale| alla quale è impostata \verb|ON UPDATE CASCADE| e \verb|ON DELETE CASCADE|, questo al fine di mantenere la base di dati in uno stato consistente; vogliamo infatti che alla cancellazione di un centro vaccinale venga eliminato anche il relativo indirizzo.
		
		\item[Registrazione]
			Ha una chiave primaria costituita da due elementi: \verb|(centro_vaccinale,CF)|, entrambi chiavi esterne.
			Il primo elemento fa riferimento al centro vaccinale in cui avviene la registrazione, meentre il secondo al codice fiscale del cittadino che si è registrato.
			Entrambi i campi sono ovviamente impostati con \verb|ON UPDATE CASCADE| e \verb|ON DELETE CASCADE| perchè l'eliminazione di uno dei due valori comporterebbe l'impossibilità di effettuare una vaccinazione nel dominio reale e un'inconsistenza logica nel dominio astratto della base di dati.
			La relazione esistente tra la tabella \verb|centro_vaccinale| e \verb|registrazione| è del tipo \emph{uno a molti}, per ogni centro vaccinale sono infatti permesse molte registrazioni;
			la relazione tra \verb|cittadino| e \verb|registrazione| è invece \emph{uno a uno}, ogni cittadino è infatti registrato presso un solo centro vaccinale.
		
		\item[Vaccinazione]
			Presenta la chiave esterna \verb|CF_citt| avente come riferimento il codice fiscale nella tabella \verb|cittadino|.
			Al campo \verb|CF_citt| sono impostati \verb|ON UPDATE CASCADE| e \verb|ON DELETE CASCADE| per le stese ragioni discusse con le tabelle precedenti.
			La relazione esistente tra \verb|cittadino| e \verb|vaccinazione| è del tipo \emph{uno a molti}, è infatti previsto il caso in cui un cittadino venga sottoposto a diverse vaccinazioni.		
		
		\item[EventoAvverso]
			Presenta una chiave esterna verso la tabella \verb|vaccinazione| nel campo \verb|ID_vaccino| a cui è associato il solito \verb|ON UPDATE CASCADE| e \verb|ON DELETE CASCADE|.
			Questa tabella tiene traccia di ogni evento avverso generatosi in seguito a una vaccinazione e registrato dal cittadino; per ogni vaccinazione possono essere registrati più eventi avversi, la relazione tra la tabella \verb|vaccinazione| e la tabella \verb|evento_avverso| è quindi del tipo \emph{uno a molti}.
	\end{description}

\subsection{Creazione tabelle}

	Riportiamo di seguito le istruzione per la creazione delle tabelle all'interno del database.

	\begin{listing}[h]
	\begin{minted}{sql}
CREATE TABLE centro_vaccinale (
	nome VARCHAR(255),
	tipologia VARCHAR(10),
	PRIMARY KEY (nome)
);
	\end{minted}
\caption{Codice per la creazione della tabella CentriVaccinali}
\end{listing}

\begin{listing}[h]
	\begin{minted}{sql}
CREATE TABLE Indirizzo (
	id_ind SERIAL PRIMARY KEY,
	identificatore VARCHAR(10),
	localizzazione VARCHAR(255),
	civico NUMERIC(4),
	comune VARCHAR(255),
	provincia CHAR(2),
	zip CHAR(5),
	centro_vaccinale VARCHAR(255),
	FOREIGN KEY (centro_vaccinale) REFERENCES centro_vaccinale(nome)
		ON UPDATE CASCADE
		ON DELETE CASCADE
);
	\end{minted}
\caption{Codice per la creazione della tabella Indirizzi}
\end{listing}

\begin{listing}[h]
	\begin{minted}{sql}
CREATE TABLE Registrazione (
	centro_vaccinale VARCHAR(255) REFERENCES centro_vaccinale
		ON UPDATE CASCADE
		ON DELETE CASCADE,
	CF CHAR(16) UNIQUE REFERENCES cittadino
		ON UPDATE CASCADE
		ON DELETE CASCADE,
	codice SERIAL,
	PRIMARY KEY (centro_vaccinale,CF)
);
	\end{minted}
\caption{Codice per la creazione della tabella Registrazione}
\end{listing}

\begin{listing}[h]
	\begin{minted}{sql}
CREATE TABLE cittadino (
	CF CHAR(16),
	nome VARCHAR(128),
	cognome VARCHAR(128),
	email VARCHAR(255),
	password VARCHAR(255),
	username VARCHAR(255) UNIQUE,
	PRIMARY KEY (CF)
);
	\end{minted}
\caption{Codice per la creazione della tabella Cittadini}
\end{listing}

\begin{listing}[h]
	\begin{minted}{sql}
CREATE TABLE Vaccinazione (
	ID_vaccino SERIAL PRIMARY KEY,
	data DATE,
	vaccino VARCHAR(20),
	CF_citt CHAR(16),
	FOREIGN KEY (CF_citt) REFERENCES Cittadino
		ON UPDATE CASCADE
		ON DELETE CASCADE
);
	\end{minted}
\caption{Codice per la creazione della tabella Vaccinazioni}
\end{listing}

\begin{listing}[h]
	\begin{minted}{sql}
CREATE TABLE EventoAvverso (
	ID_ea SERIAL PRIMARY KEY,
	evento VARCHAR(32),
	intensita SMALLINT,
	ID_vaccino NUMERIC(6),
	FOREIGN KEY (ID_vaccino) REFERENCES Vaccinazione(ID_vaccino)
		ON UPDATE CASCADE
		ON DELETE CASCADE
);
	\end{minted}
\caption{Codice per la creazione della tabella EventiAvversi}
\end{listing}